\section{Discussion}
\label{sec:opinions}

\bcp{We should think about whether to include this part.}

\subsection{A mandatory estimate of the carbon footprint by the conference organizers}

Despite a modest amount of data at our disposition and the use of a rudimentary suit of
analyses, there is no ambiguity about the relative environmental impact the choice of
location to hold a conference in has. In particular, observation~\ref{obs:dist-naive}
suggests that even setting aside any restructuration of our activities, we can hope for
saving a factor 2 by being more acute when choosing destinations. Furthermore,
observation~\ref{obs:optim} emphasizes that even a naive distribution model already gives
us material to do better, while the more ambitious perspective to model the locality effect
that we discussed through this paper would allow us for even more efficient choices.

In this light, we consider it unacceptable to continue choosing locations of conferences
either blindly, or for its scenic value. We should ponder professional relevancy with
ecological imperative. We hence formulate the following simple recommendation, that shall
have no impact on our professional activity, save for the reduction of some leisuring side
product.

\begin{recommend}
It should be made part of the mandatory process of organization of SIGPLAN
conference to estimate the carbon footprint of the options considered, and
to take the results of this analysis into account to finalize the decision.
\end{recommend}

It shall be emphasized that we do not suggest by this to consider the destination
minimizing the carbon footprint as the systematic right choice. Concerns such as
rotating over different parts of the globe or naturally accounting for availability
of qualified universities to organize should remain of major concern. We merely
assess by this recommendation the need to bring carbon footprint into the constraint
system we seek to optimize.

\subsection{A short term experiment: bi-localized or tri-localized conferences}

A considerate choice of destination to organize conferences can lead to a
non-neglectible reduction of their carbon footprint. However, a reduction of
the scale required to match by 2050 the recommendation from the Accords de Paris
will require more drastic measures. More specifically, we need to reduce the cheer
number of flights our activity induces.

There is no denying that it will have an impact on our activity, some of which will
be negative. It is hence more than ever of importance to take a reasoned approach
allowing us to balance optimization of quantitative measures, such as reducing the
carbon footprint, with qualitative imperative, such as maintaining the ideal of
an international, borderless, scientific research.

Interestingly, this novel requirement leads us to pay attention to data that may also be
relevant to our activity beyond the question of carbon footprint. These should be
taken into consideration as well while seeking a lasting restructuration of our
activity. In particular, it is implicit to assume that conferences have to be
geographically international to gather communities of researchers from all over the
world. However, observation~\ref{obs:locality} challenges strongly this intuition:
despite any claim a conference may have, the very fact that it is hold each year in
a single place on Earth rules out a vast amount of international researchers. It is most
striking with respect to programming language communities from Asia, but seems to be
true for Europeans desiring to partake in SPLASH as well for instance.

This statement is also backed up by evidence against its complement: the idea of a core
group of researchers making the essential of all editions of their favorite conference,
while not completely incorrect, is vastly overestimated. Observations~\ref{obs:old-timers}
most notably makes it very clear.

This analysis leads us to push toward strong considerations for experimenting a more
ambitious way to save carbon: giving up on the uniqueness of location of conferences and
experimenting with bi-localized or tri-localized conferences.

\begin{recommend}
  Some conferences should experiment a bi-localized format. Typically, POPL could for
  instance be held simultaneously held in Boston and Paris. A day would span over 12 hours
  instead of the usual 8. The four hours intersecting would be held simultaneously on
  both sites via visioconference. The eight other hours would be retransmitted live and
  have simple support for questions as is already put in practice.\\
  If the initial experiments go well, we recommend a progressive shift toward this
  format becoming the norm, and consideration for a third site.
\end{recommend}

We argue naturally that this change would be a truly ambitious measure to reduce
significantly the carbon footprint of conferences. But furthermore, we believe that
it would also enhance the international dimension of the conference: following
the locality effect, this would most certainly lead to an increase in participation.

A natural opposition would be to state that it is unreasonably to ask for researchers to
follow twelve hours a day of conferences, and that they would therefore miss part of the
talks. We do not deny this fact, but points out that it is already largely the case,
most conferences having two, if not three tracks in parallel.

Yet, it should not be brushed aside that this would remove some precious
physical interactions between researches from different continents. We
nonetheless argue that making these interactions the systematic default at conferences
is an historical incident. Such a restructuration of our activity would probably
be accompanied by an increase in visit to other laboratories. But that would shift
these interactions from the current situation that put hundreds of researchers in the
same building so that extremely small groups get to meet, to a more sensible ``meeting
by need" organization.

\subsection{A long term need: entirely virtualized conferences}

Bi-localized conferences strike a compromise. On the long run, research, as all
activities, shall however ambition to be entirely carbon-free. This ambitious
goal has already been embraced by some conferences\footnote{https://conference.opensimulator.org/2018/}
and seminars\footnote{https://sites.google.com/site/plustcs/}.

While the currently existing cases are either fairly experimental, or of much more
modest size than a conference such as the ones organized by SIGPLAN, they report
encouraging results. As such, we encourage experiments aiming to develop further
these techniques, and bring the cultural change they entail incrementally among
our community.

\begin{recommend}
  We recommend to conduct experiments toward the development of fully virtual
  conferences as a mean to reach a fully sustainable activity by the horizon 2050
  at the latest.
\end{recommend}
