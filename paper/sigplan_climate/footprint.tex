\section{Estimating the Footprint of Conferences}
\label{sec:footprint}

Carbon footprint is the essential metric that we seek to reduce.
Accordingly, it is also the starting point of our analysis.  We introduce in
this section the methodology we used and tool we built to conduct all of our
analyses, and we describe the first results from our dataset.

\subsection{Methodology for Evaluating Carbon Cost}
\label{sec:methodo}

We conduct all our analyses through a \python{} script, publicly available at
\url{https://github.com/YaZko/acm-climate}. We describe its behavior and give
a brief overview of its use in Section~\ref{sec:software}.
\bcp{Maybe this repo needs a more
informative name?}

We make the following assumptions:
\begin{itemize}
\item we assume that participant travel accounts for the entire carbon
footprint of a conference;
\item we assume that \emph{all} participants travel by plane, in economy
class; 
\item we assume that the airports in the conference city and in each
participant's home city are close enough to the actual end points of their
travel for their locations to be assimilated;
\item we assume that all flights are direct;
\item we assume that the geodesic distance is the one taken by planes.
\end{itemize}
Estimating the errors introduced by these assumptions and refining the
analysis to make more realistic assumptions would obviously be very
valuable! 
%
For this first effort, we are mainly aiming to get a 
{\em relative} evaluation of different potential strategies for reducing
footprints; for this purpose, we believe these assumptions are good enough. 

The distance traveled by each participant is converted to an amount of
emissions expressed in \gaz. To do this conversion, we use a standard model
introduced as part of the \texttt{DEFRA 16} report on Greenhouse gas
\footnote{\url{https://www.gov.uk/government/publications/greenhouse-gas-reporting-conversion-factors-2016}}
\footnote{\url{https://co2calculator.acm.org/methodology.pdf}} conducted by
the British Government.

The model distinguishes three classes of flight, depending on their length:
short, medium, and long haul. Each category is associated with a linear
coefficient relating the distance of travel to the amount of \gaz
emitted.  

A second linear coefficient, identical for all flights, is the so-called
\emph{radiative forcing index}; this is used to account for the difference
in radiative forcing between the same emissions at ground level compared to
high in the atmosphere.  We use the value $1.891$ for this coefficient, as
suggested by R. Sausen et al.~\cite{Sausen05}

We thus obtain the following piecewise-linear model of emissions for a
flight covering $d$ kms: 

\begin{center}
\gazunit \quad=\quad
\begin{tabular}{@{}lll}
$1.891 * 0.14735 * d$ & if $d < 785$ \\
$1.891 * 0.08728 * d$ & if $785 \leq d < 3700$ \\
$1.891 * 0.077610 * d$ & if $3700\leq d  $ 
\end{tabular}
\end{center}
% \begin{itemize}
% \item $1.8& * 0.14735 * d$ \gazunit if $d < 785$ 
% \item $1.891 * 0.08728 * d$ \gazunit if $785 \leq d < 3700$
% \item $1.891 * 0.077610 * d$ \gazunit if $3700\leq d  $
% \end{itemize}

%% It should be noted that experiments with other models show significant variance
%% in absolute value, but resilience in relative values.\bcp{Maybe worth
%%   showing some numbers justifying these statements?}\yz{I agree, will
%%   do}\bcp{Assuming that we can get our numbers to agree with CoolEffect's,
%%   we could also mention this!} Once again, refining the
%% model would hence be a valuable work, but using this simple standard and
%% well-established one appears appropriate to draw conclusion in terms of
%% \emph{relative} impact of different measures.

%% This first pass of the script therefore give us an estimation of the footprint
%% of our conferences. We have implemented on top of it several analyses aiming to
%% estimate the correlation some concrete factors upon which conference organizers
%% can act may have with this footprint.
%% The description of these analyses will cover Section~\ref{sec:community} to \ref{sec:speculate}.

\subsection{Conference Footprints}

\begin{table}
\begin{tabular}{|l|l|c|c|c|}
  \hline%
  \bfseries Event & \bfseries Location & \bfseries \# Participants & \bfseries Total cost & \bfseries Average cost 
\csvreader[head to column names]{../../output/sigplan/footprint_confs.csv}{}%
{\\\conf\ \year & \location & \csvcoliv & \csvcolv & \csvcolvi}%
\\\hline
\end{tabular}
\caption{For each \event: location, number of participants and carbon cost, total and average per participant, in \gazunitbis,}
\label{table:footprint}
\end{table}

We now turn to the estimation of the footprint of our dataset.
Table~\ref{table:footprint} depicts the total and average carbon cost per participant of
all conferences analyzed. This cost is estimated in terms of \gazunitbis{}
(metric tons of CO$_2$-equivalent) of emissions.
The main data of interest is arguably the last column depicting the average cost per participant.

The lowest average per-participant cost of our dataset is PLDI'18 at
0.9\gazunitbis, while the highest one is ICFP'16 at 1.93\gazunitbis.

\begin{obs}
The average per participant carbon footprint of conferences due to air
travel varies from one to another by up to a factor of 2.
\label{obs:footprint}
\end{obs}

